\documentclass{svproc}
\usepackage{url}
\tolerance=1
\emergencystretch=\maxdimen
\hyphenpenalty=10000
\usepackage{caption}
\usepackage{float}
\def\UrlFont{\rmfamily}

\begin{document}
\mainmatter
\title{Your Project Report Title Here
}
\subtitle{CS7IS2 Project (2019/2020)}
\author{Kulkarni Shravani Deepak, Kapoor Shuchita, Asolkar Jayprakash \and Chauhan Paritosh}

\institute{
\email{email\_address\_1}, (multiple email addresses separated by comma)}

\maketitle              % typeset the title of the contribution

\begin{abstract}
The abstract should summarize the contents of the report and should contain at least 70 and at most 150 words. It should be set in 9-point font size and should be inset 1.0 cm from the right and left margins. There should be two blank (10-point) lines before and after the abstract. This document is in the required format. The abstract should give a concise overview of the main points of the report: the motivation behind the work, a very high level description of the problem and how it was solved by the proposed algorithms. The abstract must not include any figures or table.
\keywords{computational geometry, graph theory, Hamilton cycles}
\end{abstract}
%

This document is a guideline for writing the final report for the CS7IS2 module Artificial Intelligence. You should follow its general structure as shown below.
You should not change its format (font, size, margin, space, etc.). 
Your report should be between five and ten pages. Report that not comply to the format or exceed the maximum length will be penalised (-5 marks).
Brevity is desirable in communication, however you should provide all those details necessary for the good understanding of the described methods and algorithms. 
The report will be graded on the basis of:
\begin{itemize}
\item Originality;
\item Technical soundness;
\item Organisation;
\item Clarity of presentation;
\item Adequacy of bibliography/Results (this last point strongly depends on the type of report)
\end{itemize}


Your report should provide a survey and an experimental comparison of multiple solution approaches to a particular problem. This is a critical review of at least three papers that significantly contributed to advance the state-of-the-art for the problem you are analysing. It should not be a mere summary of the papers. You are expected to conduct an analytical review of the methods under analysis to try to find common aspect and differences, connections between methods, drawbacks and open problems. Unless the faced problem has emerged recently, students should choose their papers by diversifying the range of approaches used to solve the problem. A good guideline could be to choose a paper from a decade or two ago, and a couple of more recent papers. You need to experimentally evaluate approaches in a simulation of a problem, in a range of scenarios, and analyse the pros and cons of each approach. 

\section{Introduction}
There have been many heuristic searching algorithms in the AI domain. Many gaming problems have been formed to test their efficiency. The N-puzzle problem is among these problems, and it is also included in the list of classic difficulty problems. This puzzle consists of N square tiles with tiles numbered from 1 to N. The goal of this problem is to rearrange the numbers in ascending order on the MxM board. This paper focusses on the 8-puzzle problem, which is a smaller version of the N-puzzle problem, and finds solutions using four algorithms (BFS, DFS, A*).  Also, these solutions are compared by examing their evaluation metrics, such as the cost of the path and time taken to find the solution. \\

\noindent The paper is structured as follows: section 2 discusses prior literature outlining the algorithms used in those papers,  section 3 defines the 8-puzzle problem in detail and describes the algorithms used, section 4 shows the results and comparisons of these algorithms.

\section{Related Work}
In this section you will discuss possible approaches to solve the problem you are addressing, justifying your choice of the 3 you have selected to evaluate. Also, briefly introduce the approaches you are evaluating with a specific emphasis on differences and similarities to the proposed approach(es).

\section{Problem Definition and Algorithm}
The n-puzzle problem is a classic AI problem which contains n numbered tiles and one blank tile. These plates are arranged in a grid with the numbered tiles marked as 1, 2, 3 .. n. The blank tile is movable in the grid which can be achieved by moving the numbered tiles in the place of empty tile. To achieve the goal state of the puzzle, the grid should be arranged in such a way that the empty tile is at the first position and all the numbered tiles are arranged in numerical order. The problem in solving the n-puzzle is to reach the goal state in as little cost as possible. For instance, an eight-puzzle problem consists of 8 numbered tiles - 1, 2, 3, 4, 5, 6, 7, 8. The goal state of the 8-puzzle problem is shown in the figure. \\

The cost to achieve the goal can be measured in terms of the number of movements done. At any given intermediate state, there can be at least two and at most four possible movements.

\subsection{Subsection Title}

\section{Experimental Results}
This section should provide the details of the evaluation. Specifically:
\begin{itemize}
\item Methodology: describe the evaluation criteria, the data used during the evaluation, and the methodology followed to perform the evaluation. 
\item Results: present the results of the experimental evaluation. Graphical data and tables are two common ways to present the results. Also, a comparison with a baseline should be provided.
\item Discussion: discuss the implication of the results of the proposed algorithms/models. What are the weakness/strengths of the method(s) compared with the other methods/baseline?
\end{itemize}

\section{Conclusions}
Provide a final discussion of the main results and conclusions of the report. Comment on the lesson learnt and possible improvements.


A standard and well formatted bibliography of papers cited in the report. For example:

\begin{thebibliography}{6}
%

\bibitem {smit:wat}
Smith, T.F., Waterman, M.S.: Identification of common molecular subsequences.
J. Mol. Biol. 147, 195?197 (1981). \url{doi:10.1016/0022-2836(81)90087-5}

\bibitem {may:ehr:stein}
May, P., Ehrlich, H.-C., Steinke, T.: ZIB structure prediction pipeline:
composing a complex biological workflow through web services.
In: Nagel, W.E., Walter, W.V., Lehner, W. (eds.) Euro-Par 2006.
LNCS, vol. 4128, pp. 1148?1158. Springer, Heidelberg (2006).
\url{doi:10.1007/11823285_121}

\bibitem {fost:kes}
Foster, I., Kesselman, C.: The Grid: Blueprint for a New Computing Infrastructure.
Morgan Kaufmann, San Francisco (1999)

\bibitem {czaj:fitz}
Czajkowski, K., Fitzgerald, S., Foster, I., Kesselman, C.: Grid information services
for distributed resource sharing. In: 10th IEEE International Symposium
on High Performance Distributed Computing, pp. 181?184. IEEE Press, New York (2001).
\url{doi: 10.1109/HPDC.2001.945188}

\bibitem {fo:kes:nic:tue}
Foster, I., Kesselman, C., Nick, J., Tuecke, S.: The physiology of the grid: an open grid services architecture for distributed systems integration. Technical report, Global Grid
Forum (2002)

\bibitem {onlyurl}
National Center for Biotechnology Information. \url{http://www.ncbi.nlm.nih.gov}


\end{thebibliography}
\end{document}
